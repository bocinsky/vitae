\section{Employment}

{\bf University of Montana}, Missoula, Montana
\begin{list1}

\item[] January 2021—present\hspace{.2cm}Director of Climate Extension, Montana Climate Office
\begin{list2}
\item[] Currently manages a >\$2 million research and extension grant portfolio in collaboration with Tribal nations, researchers, and federal and state agencies; builds relationships with users of climate data and information in Montana, including outreach supporting agriculture, forestry, recreation, urban and rural climate resilience planning, and other climate-sensitive issues; develops partnerships with Native nations in Montana in support of their needs as related to climate and drought monitoring and education; conducts independent and collaborative research broadly surrounding topics relating to human-environment relationships in the past, present, and future.
\end{list2}

\vspace{0.1in}

\item[] January 2018—January 2021\hspace{.2cm}Research Associate, Montana Climate Office
\begin{list2}
\item[] Co-project director and climate science lead for the \emph{Montana Drought \& Climate} project, funded by the USDA; developed web dashboards and print newsletters aimed at better communicating climate information to farmers and ranchers in Montana; improved the drought readiness and resilience of the Montana agricultural industry.
\end{list2}
\end{list1}

{\bf Crow Canyon Archaeological Center}, Cortez, Colorado
\begin{list1}

\item[] August 2018—January 2021\hspace{.2cm}Director, Research Institute
\begin{list2}
\item[] Managed a roughly >\$600,000 research portfolio funded through grants, endowments, and private donations; helped lead a financial turnaround of the Center and navigate the Covid-19 pandemic; and launched Crow Canyon’s postdoctoral fellowship program; developed leadership in non-profit administration, project management, and institutional advancement.
\end{list2}

\vspace{0.1in}

\item[] January 2016—December 2016\hspace{.2cm}Director of Sponsored Projects
\begin{list2}
\item[] Developed research collaborations between Crow Canyon archaeologists and external researchers; administered a >\$1 million portfolio of grant-funded research projects; conducted cutting-edge interdisciplinary research on ancient agricultural practices, climate change, and global sustainability.
\end{list2}

%\vspace{0.1in}
%
%\item[] Summer 2012\hspace{.2cm}Field Intern, Basketmaker Communities Project
%\begin{list2}
%\item[] Excavation and education at the Dillard Site, a Basketmaker III community; Gave public talks on Pueblo prehistory to visitors of all ages. Directed by Shanna Diederichs.
%\end{list2}
\end{list1}

{\bf Washington State University}, Pullman, Washington
\begin{list1}
\item[] 2015\hspace{.2cm}Post-doctoral Researcher, \emph{Synthesized Knowledge of Past Environments}
\begin{list2}
\item[] Worked to bring paleoenvironmental data to scientists and the general public; collaborated on the development of web mapping services; integrated environmental data with cultural data to gauge impacts of climate change on humans. Directed by Timothy A Kohler.
\end{list2}

\vspace{0.1in}

\item[] Aug 2008–Dec 2014\hspace{.2cm}Research Fellow, Department of Anthropology
\begin{list2}
\item[] Agent based modeling with the Village Ecodynamics Project; documented and debugged code and expanded study areas; coordinated data storage and delivery among 4 institutions and 16 researchers. Directed by Timothy A Kohler.
\end{list2}

%\vspace{0.1in}
%
%\item[] Fall 2014, Spring 2013, Spring 2012\hspace{.2cm}Instructor, Department of Anthropology
%\begin{list2}
%\item[] Developed and taught three courses at the undergraduate level; 15–90 students.
%\end{list2}
%
%\vspace{0.1in}
%
%\item[] Aug 2014–Dec 2014\hspace{.2cm}Instructor, \emph{ANTH 101: General Anthropology}, 90 students
%\begin{list2}
%\item[] An introduction to the four fields of anthropology
%\end{list2}
%
%\vspace{0.1in}
%
%\item[] Jan 2013–May 2013\hspace{.2cm}Instructor, \emph{ANTH 490: Themes in Anthropology}, 15 students
%\begin{list2}
%\item[] The capstone seminar in the anthropology major at WSU; a review of important themes in anthropological thought and several controversial issues in anthropology
%\end{list2}
%
%\vspace{0.1in}
%
%\item[] Jan 2012–May 2012\hspace{.2cm}Instructor, \emph{ANTH 331: The Americas Before Columbus}, 40 students 
%\begin{list2}
%\item[] A review course covering North and South American prehistory, and emphasizing model-based approaches in archaeology
%\end{list2}
\end{list1}

%{\bf Mesa Verde National Park}, Colorado
%\begin{list1}
%\item[] Summers 2009, 2011, 2012\hspace{.2cm}Field Technician, The Mesa Verde Community Center Survey
%\begin{list2}
%\item[] Visited and documented large aggregated villages in MVNP; developed iPad-based site-recording workflows integrated with mapping tools in AutoCAD and Adobe Illustrator; Drafted survey final report in \LaTeX. Directed by Donna M. Glowacki.
%\end{list2}
%\end{list1}
%
%{\bf Field Museum of Natural History}, Chicago, Illinois
%\begin{list1}
%\item[] Summer 2008\hspace{.2cm}Collections Intern
%\begin{list2}
%\item[] Facilitated access to collections for visiting researchers, analyzed ceramics, and constructed protective housing for ceramics. Directed by Scott Demel.
%\end{list2}
%\end{list1}

%{\bf University of Notre Dame}, Notre Dame, Indiana
%\begin{list1}
%\item[] Aug 2007–May 2008\hspace{.2cm}Research Assistant, Department of Anthropology
%\begin{list2}
%\item[] Stable isotope analysis of faunal and human remains. Directed by Mark Schurr
%\end{list2}
%\end{list1}

%{\bf University of Notre Dame}, Notre Dame, Indiana
%\begin{list1}
%\item[]
%\begin{tabular}{@{}p{1.25in}p{4in}}
%Aug 2007–May 2008  & Research Assistant, Department of Anthropology \\
%Aug 2005–May 2008  & Tour Guide, University Admissions \\
%Aug 2005–May 2008  & Office Assistant, Glynn Family Honors Program \\
%Aug 2004–May 2005  & Office Assistant, Mendoza College of Business \\
%\end{tabular}
%\end{list1}

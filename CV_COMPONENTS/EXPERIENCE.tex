\section{Experience \& \\ Achievements}


{\bf Grant Achievements}
\begin{bullet1}
\item Project director or Co-PD for a portfolio of funded projects with budgets ranging from \$35,000 to \$1.5 million.
\item Co-PI on \emph{Developing and Deploying SKOPE---A resource for Synthesizing Knowledge of Past Environments}, funded through a National Science Foundation special competition, \emph{Resource Implementations for Data Intensive Research in the Social Behavioral and Economic Sciences}(\href{http://www.nsf.gov/awardsearch/showAward?AWD_ID=1637171}{SMA-1347973}). Part of a collaborative research grant between Washington State University, Arizona State University, and the University of Illinois at Urbana-Champaign. \$254,189 awarded to WSU/CCAC, \$1,339,658 total.
\item Secured \$180,000 in competitive masters and doctoral funding from the National Science Foundation and Washington State University resources, plus full tuition waivers from WSU for the duration of graduate school.
\item Secured \$16,800 in research funding from the National Science Foundation in support of travel and materials.
\end{bullet1}


{\bf Research Collaborations}
\begin{bullet1}
\item Extensive experience collaborating with large, interdisciplinary teams of researchers from academic, public, and private sectors
\item Managed and coordinated data acquisition, storage, and delivery between team members using Subversion and Git repositories with local and cloud-based storage
\item Authored 30 peer-reviewed research articles published in journals including \emph{Science}, \emph{Nature Communications},  \emph{Science Advances}, and \emph{American Antiquity}, ten book chapters, five technical reports, and four  papers
\end{bullet1}


{\bf Software Development}
\begin{bullet1}
\item Developer of several packages in \emph{R}, including \emph{FedData} for downloading and processing geospatial and climate data from federated data sources
\item Packages are being adopted by academic researchers (at Washington State University and Arizona State University) and in the public sector (Natural Resources Conservation Service); \emph{FedData} downloaded over 79,000 times since release
\item Lead developer in the \emph{Village Ecodynamics Project}---built agent-based geospatial simulations of ancient Pueblo human-environment interaction using the \emph{RePAST} simulation framework (in the Java computer language)
\end{bullet1}
%
%
%{\bf Classroom Management}
%\begin{bullet1}
%\item Developed curricula and taught three courses at introductory and advanced levels, and managed class sizes ranging between 15 and 90 students
%\item Emphasized a science-based comparative approach to anthropological research
%\item Created collaborative assignments designed to enhance leadership, management, and presentation skills among students
%\item Challenged students to apply anthropological perspectives to real-world problems
%\end{bullet1}